% !TEX TS-program = xelatex
\documentclass[9pt]{developercv}

\usepackage{romannum}
\usepackage{phonenumbers}

\begin{document}


%----------------------------------------------------------------------------------------
%	TITLE AND CONTACT INFORMATION
%----------------------------------------------------------------------------------------

\begin{minipage}[t]{0.45\textwidth}
	\vspace{-\baselineskip}
	
	\colorbox{black}{{\Huge\textcolor{white}{\textbf{\MakeUppercase{Alessio}}}}}

	\colorbox{black}{{\Huge\textcolor{white}{\textbf{\MakeUppercase{Coltellacci}}}}}
	
	\vspace{6pt}
	
	{\huge System developer}
\end{minipage}
\begin{minipage}[t]{0.275\textwidth}
	\vspace{-\baselineskip}
	\icon{MapMarker}{12}{Toulouse, France}\\
	\icon{Phone}{12}{\phonenumber[country=FR]{+33645133174}}\\
	\icon{At}{12}{\href{mailto:lightplay8@gmail.com}{lightplay8@gmail.com}}\\
\end{minipage}
\begin{minipage}[t]{0.275\textwidth}
	\vspace{-\baselineskip}
	
	\icon{Github}{12}{\href{https://github.com/NotBad4U}{github.com/NotBad4U}}\\
	\icon{Twitter}{12}{\href{https://twitter.com/@lightplay8}{@lightplay8}}\\
\end{minipage}

\vspace{0.5cm}

%----------------------------------------------------------------------------------------
%	INTRODUCTION, SKILLS AND TECHNOLOGIES
%----------------------------------------------------------------------------------------

\cvsect{Who Am I?}

\begin{minipage}[t]{0.4\textwidth}
	\vspace{-\baselineskip}
	I have been working in the Cloud industry for two
	years now, with a focus on systems, virtualization and
	networking. I’m always hungry to learn new things
	and I’m happiest when I can marry math, computer
	science, and philosophy on a problem and create
	novel solutions. It is not in science that I found
	happiness, but in the acquisition and the discovery of
	it. I am a lateral thinker geared towards solving
	complex and Kafkaïen problems by trying to meet
	them with the simplest system that can be found,
	which is for me to find the most elegant, constricted
	and synthesized solution. Furthermore, I have a keen
	desire to try out programming, algorithmic concepts
	and languages on a wider spectrum through
	experiments and research.
\end{minipage}
\hfill
\begin{minipage}[t]{0.5\textwidth}
	\vspace{-\baselineskip}
	\begin{barchart}{5.5}
		\baritem{Rust}{100}
		\baritem{C}{100}
		\baritem{Java/JavaScript/Go}{70}
		\baritem{Latex}{50}
		\baritem{Scala}{40}
		\baritem{NASM/ARM}{35}
		\baritem{PDDL}{35}
		\baritem{Coq}{25}
		\baritem{TLA+}{25}
		\baritem{eBPF}{25}
	\end{barchart}
\end{minipage}

\begin{center}
	\bubbles{5/Linux, 5/Docker, 6/Git, 5/KVM\&Qemu, 4/Emacs, 3/VSCode, 3/Blender}
\end{center}

%----------------------------------------------------------------------------------------
%	EDUCATION
%----------------------------------------------------------------------------------------

\cvsect{Education}

\begin{entrylist}
	\entry
		{2016 -- 2018}
		{Master's degree}
		{\href{http://www.univ-tlse3.fr/}{Paul Sabatier university, Toulouse}}
		{A formation in Software development in order to become an expert in software engineering, with a
		particular focus on distributed systems and critical software. Skills I have acquired include: Coq, TLA
		+, EventB, and UPPAAL.}
	\entry
		{2014 -- 2016}
		{Bachelor's degree}
		{\href{http://www.univ-tlse3.fr/}{Paul Sabatier university, Toulouse}}
		{A formation in computer science and fundamental maths and physics in order to generate enthusiasm
		for science and research. With initiation and experiments on the work as a research scientist.}
	\entry
		{2013}
		{Highschool degree}
		{\href{https://pierre-paul-riquet.mon-ent-occitanie.fr/lycee-riquet/}{Highschool Pierre-Paul Riquet, Toulouse}}
		{With an option in computer science and finalist at the regional and national engineering science Olympiad,
		where we participated with our self-made biped robot.}
\end{entrylist}

%----------------------------------------------------------------------------------------
%	EXPERIENCE
%----------------------------------------------------------------------------------------

\cvsect{Experience}

\begin{entrylist}
	\entry
		{2017 -- 2019}
		{System Developer}
		{Clever Cloud.}
		{
			Clever Cloud is a platform as a service (PAAS) following the immutable infrastructure paradigm. So
			servers are never modified after they’re deployed. If something needs to be updated or fixed in any way,
			new servers built with the appropriate changes are provisioned to replace the old ones. This paradigm
			doesn’t fit well with the current reverse proxy and DNS server which doesn’t support high reload of their
			configuration. I contributed to the development of the OS project which solves this problem:\\
			- Stream-DNS, an authority and resolver DNS server with an event-sourcing architecture to manage his DNS zones.\\
			- Sozu, an HTTP reverse proxy, configurable at runtime, fast and safe, built-in Rust.\\
			\textbf{\texttt{Rust}}\slashsep\texttt{Scala}\slashsep\texttt{Webassembly}\slashsep\texttt{Docker}\slashsep\texttt{Network}\slashsep\texttt{KVM\&Qemu}\slashsep\texttt{Microservices}
		}
	\entry
		{2016 -- 2018}
		{Freelance}
		{}
		{
			Alongside my studies, I worked as freelance in small, medium and large companies. In order to improve
			my skills in software engineering. I worked on topics ranging from web development and distributed
			systems for aeronautics in the Aircraft Communication Addressing and Reporting System (ACARS).\\
			\texttt{Angular}\slashsep\texttt{NodeJS}\slashsep\texttt{Docker}\slashsep\texttt{MongoDB}\slashsep\texttt{Elasticsearch}
		}
	\entry
		{April -- July 2017}
		{Internship}
		{Fujitsu systems Europe.}
		{
			I contributed to the grid computing of Fujitsu, by adding the integration of WebDAV protocol for the
			exchange of batch jobs files between nodes in the grid.\\
			\texttt{JavaEE}\slashsep\texttt{WebDAV}\slashsep\texttt{Jenkins}\slashsep\texttt{Docker}\slashsep\texttt{MongoDB}
		}
	\entry
		{April -- Sep. 2015}
		{Internship}
		{Ekito}
		{
			I participated to the benchmark of new surf drifts made by the surfer of S-Wings company. I transformed
			surfboards in a connected surfboard by putting IoT devices on it. I developed the algorithm and driver
			of the devices to extract the data of the surfboard (speed, acceleration, etc.) and transfer it through
			radio in 433 Hz. This data was post treated on a software that I made, with a 3D view of the surfboard
			to replay the course through data extracted.\\
			\texttt{IOT}\slashsep\texttt{C}\slashsep\texttt{NodeJS}\slashsep\texttt{WebGL}
		}
\end{entrylist}

%----------------------------------------------------------------------------------------
%	CONFERENCES
%----------------------------------------------------------------------------------------

\cvsect{Speaker in conferences}
\begin{itemize}
	\item Devoxx United Kingdom 2019, Introduction to Rust programming language.
	\item Devoxx 2018 Ukraine, Actor, an elegant model for concurrent and distributed computation.
	\item Voxxed Days Luxembourg 2018, Understand the Linux container by building one from scratch.
	\item Devoxx 2018 France, Introduction to Rust programming language.
\end{itemize}

%----------------------------------------------------------------------------------------
%	RESEARCH INTERESTS
%----------------------------------------------------------------------------------------

\cvsect{Research Interests}
\begin{entrylist}
	\entry
	{}
	{Temporal Logic}
	{}
	{
		Cover approaches to representation and reasoning about time, events and temporal information within a
		logical framework. Narrowly to refer specifically the modal-logic type of approach introduced by Arthur
		Prior under the name of Tense Logic.\\
		\texttt{Logic Tense}\slashsep\texttt{LTLP}\slashsep\texttt{CTL*}\slashsep\texttt{TLA+}\slashsep\texttt{UPPAAL}\slashsep\texttt{PDDL 3}
	}
	\entry
	{}
	{Modal Logic}
	{}
	{
		The study within a logical framework of the deductive behavior in related systems, which include the
		logic for belief, tense, deontology, geography and many others. It’s particularly valuable in the formal
		analysis of specifications and philosophical arguments.\\
		\texttt{Kripke}\slashsep\texttt{Deontic logic}
	}
	\entry
	{}
	{Actor model}
	{}
	{
		A message passing concurrency model that was originally proposed by Hewitt in 1973. This model and
		his plethora variations are valuable to avoid common concurrency issues such as low-Level data races
		and deadlock by design, due to the nature of actors: a strictly isolated entity and his nature of its
		communication mechanism.\\
		\texttt{Sessions-type}\slashsep\texttt{Akka}\slashsep\texttt{Actix}
	}
\end{entrylist}

%----------------------------------------------------------------------------------------
%	ADDITIONAL INFORMATION
%----------------------------------------------------------------------------------------
\begin{center}

\begin{minipage}[t]{0.3\textwidth}
	\vspace{-\baselineskip}
	\cvsect{Languages}
	\begin{itemize}

	\item \textbf{French} - Mother tongue
	\item \textbf{English} - Level C1 (CECRL)
	\end{itemize}

\end{minipage}
\begin{minipage}[t]{0.3\textwidth}
	\vspace{-\baselineskip}
	
	\cvsect{Hobbies}
	\begin{itemize}
		\item[\faDribbble] Basket-ball, university player.
		\item[\faPaintBrush] 3D computer graphics with Blender and Photoshop.
		\item[\faBook] Litterature of \Romannum{18}\textsuperscript{e} - \Romannum{19}\textsuperscript{e}.
		\item[\faBook] Philosophy of the \Romannum{18}\textsuperscript{e}.
	\end{itemize}
\end{minipage}
\end{center}

%----------------------------------------------------------------------------------------

\end{document}
